
\maketitle

\noindent\textbf{Abstract.} Self-organizing maps usually rely on a
predetermined topology of the neural space (the map), which is either
a rectangular or a hexagonal Cartesian grid. When the intrinsic dimension of 
the input space is much higher than the allowed dimension of the neural space,
then the self-organizing map can be ill-formed. To overcome this problem in 
high dimensional input spaces we propose a variation of the self organizing map
algorithm, where we consider random placement of neurons on a high-dimensional
manifold. The positions of the neural units are drawn from a blue noise 
distribution thus various topologies can be derived. These topologies possess 
random (but controllable) discontinuities that allow for a more flexible s
elf-organization, especially with high-dimensional data. The proposed algorithm
has been tested on one-, two- and \gid[X]-dimensions tasks as well as on 
reorganization problems, where we either removed or added neurons to the map.
\gid{TOOLS WE USED TO ASSESS THE QUALITY AND PERFORMANCE}.\\


\noindent\textbf{Keywords.} Self Organization, Neural Networks, 
Vector Quantization, Voronoi Tesselation, Neural Map Topology.

\bigskip
